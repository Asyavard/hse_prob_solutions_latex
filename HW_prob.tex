\documentclass[112pt, cmcyralt]{article}
\usepackage[utf8]{inputenc}
\usepackage[english,russian]{babel}
\usepackage{amsmath,amsfonts,amssymb,mathtools,amsthm}
\usepackage[paper=a4paper, top=15mm, bottom=13.5mm, left=13mm, right=13mm, includefoot]{geometry}
\usepackage{pgfplots} %графики

\DeclareMathOperator{\Lin}{\mathrm{Lin}}
\DeclareMathOperator{\Linp}{\Lin^{\perp}}
\DeclareMathOperator*\plim{plim}
\DeclareMathOperator{\grad}{grad}
\DeclareMathOperator{\card}{card}
\DeclareMathOperator{\sgn}{sign}
\DeclareMathOperator{\sign}{sign}

\DeclareMathOperator*{\argmin}{arg\,min}
\DeclareMathOperator*{\argmax}{arg\,max}
\DeclareMathOperator*{\amn}{arg\,min}
\DeclareMathOperator*{\amx}{arg\,max}
\DeclareMathOperator{\cov}{Cov}
\DeclareMathOperator{\Var}{Var}
\DeclareMathOperator{\Cov}{Cov}
\DeclareMathOperator{\Corr}{Corr}
\DeclareMathOperator{\pCorr}{pCorr}
\DeclareMathOperator{\E}{\mathbb{E}}
\let\P\relax
\DeclareMathOperator{\P}{\mathbb{P}}

\usepackage{tikz, pgfplots} % язык для рисования графики из latex'a
\usetikzlibrary{trees} % прибамбас в нем для рисовки деревьев
\usetikzlibrary{arrows} % прибамбас в нем для рисовки стрелочек подлиннее
\usepackage{tikz-qtree} % прибамбас в нем для рисовки деревьев

\newcommand{\cN}{\mathcal{N}}
\newcommand{\cU}{\mathcal{U}}
\newcommand{\cBinom}{\mathcal{Binom}}
\newcommand{\cBin}{\cBinom}
\newcommand{\cPois}{\mathcal{Pois}}
\newcommand{\cBeta}{\mathcal{Beta}}
\newcommand{\cGamma}{\mathcal{Gamma}}

\newcommand \R{\mathbb{R}}
\newcommand \N{\mathbb{N}}
\newcommand \Z{\mathbb{Z}}

%%%%%%%%%%%%%%%%%% вставки
%%%%%%%%%%%%%%%%%%%%%%%%%%%%%%%%%%%%%%% Списки без уродских отступов
\newenvironment{enumerate*}{
\begin{enumerate}
  \setlength{\itemsep}{0pt}
  \setlength{\parskip}{0pt}
  \setlength{\parsep}{0pt}
}{\end{enumerate}}

\newenvironment{itemize*}{
\begin{itemize}
  \setlength{\itemsep}{0pt}
  \setlength{\parskip}{0pt}
  \setlength{\parsep}{0pt}
}{\end{itemize}}

\abovedisplayskip=0mm
\abovedisplayshortskip=0mm
\belowdisplayskip=0mm
\belowdisplayshortskip=0mm

\newcommand{\dx}[1]{\,\mathrm{d}#1} % для интеграла: маленький отступ и прямая d
\newcommand{\ind}[1]{\mathbbm{1}_{\{#1\}}} % Индикатор события
%\renewcommand{\to}{\rightarrow}
\newcommand{\eqdef}{\mathrel{\stackrel{\rm def}=}}
\newcommand{\iid}{\mathrel{\stackrel{\rm i.\,i.\,d.}\sim}}
\newcommand{\const}{\mathrm{const}}

\def\be{\hangindent=14mm \hangafter=1 \noindent}



\title{Подборка контрольных работ и экзаменов по теории вероятностей}
\author{Вардикян Ася, Кабанова Юля, Петрова Дарина \\ БЭК182}

\begin{document}
\maketitle
\newpage
\begin{enumerate}

\section{Промежуточный экзамен 2016-2017 гг}

%1
\item
Вероятность того, что первой вытянутой картой окажется тройка, равна: $\P(A)=\frac{4}{52}=\frac{1}{13}$\\
Безусловная вероятность события В считается по формуле:
$$\P(B)=\frac{A_4^1\cdot A_3^1+A_{48}^1\cdot A_4^1}{A_{52}^2}=\frac{4\cdot3 + 48\cdot4}{2652} = \frac{17}{221},$$\\
где $A_{52}^2$ – общее число исходов испытания,\\
$A_4^1$ – число исходов, когда вытянута одна из 4 семерок (первый раз вытянута не семерка),
\\$A_3^1$ – число исходов, когда во второй раз вытянута одна из трех оставшихся семерок (первой вытянутой картой была семерка), \\$A_{48}^1$ - число исходов, когда вытянута первый раз вытянута не семерка.\\

Если А произошло, то перед извлечением второй карты останется 51 карта, из которых 4 являются семерками, следовательно, $\P(B|A)=\frac{4}{51}$.

Так как $\P(B|A)=\frac{4}{51} \neq \frac{17}{221}=\P(B)$, то события А и B являются зависимыми.\\

Аналогично, В и С являются зависимыми:\\
безусловная вероятность события С (первый и второй раз вытянута не дама пик, третий раз - дама пик) считается по формуле:\\
$$\P(C)=\frac{51}{52}\cdot\frac{50}{51}\cdot\frac{1}{5}=\frac{1}{52}$$\\
Вероятность того, что событие С произошло при условии, что В произошло (при этом первый раз должна быть вытянута не дама пик):\\
$$\P(C|B)=\frac{51}{52}\cdot\frac{17}{221}\cdot\frac{1}{5}=\frac{867}{574600}\neq\P(C)=\frac{1}{52}$$ следовательно, С и В зависимы.\\

\textbf{Ответ: B}

%2
\item
Согласно свойствам функции плотности, она:\\

1)	Больше или равна нулю, следовательно, А и В неверно.\\
2) $\int_{-\infty}^{+\infty} f_X(x)dx=1$\\
Так как в С:$\int_{1}^{+\infty} \frac{1}{x^2}dx=\frac{-1}{x}$|_1^\infty=1$, 
то $f_X(x) =$ 
 \begin{cases}
   \frac{1}{x^2}, &\text{$x \in [1;+\infty)$}\\
   0, &\text{иначе}$\\
 \end{cases}
может являться функцией плотности.\\
Так как в E:$\int_{0}^{2}x^2dx=\frac{x^3}{3}|_0^2=\frac{8}{3}\neq 1,$ следовательно, $f_X(x)=$ 
 \begin{cases}
   x^2, &\text{$x \in [0;2]$}\\
   0, &\text{иначе}\\
 \end{cases}
не может являться функцией плотности (Е неверно).\\

$f_X(x) =\frac{1}{\sqrt{2\pi}}e^{-2x}$ не может являться функцией плотности, так как $\int_{-\infty}^{+\infty} \frac{1}{\sqrt{2\pi}}e^{-2x}dx=\frac{1}{\sqrt{2}}\neq1$ (D неверно).\\

\textbf{Ответ: С}

%3
\item 
По свойству $\Cov(X,Y)=\E(XY)-\E(X)\E(Y)$ получается: 
$\E(XY)=\E(X)\E(Y)+\Cov(X,Y)=3\cdot2+2=8$\\

\textbf{Ответ: A}

%4
\item
$\Corr(X,Y)= \frac{\Cov(X,Y)}{\sqrt{\Var(X)\cdot \Var(Y)}}=\frac{2}{\sqrt{12\cdot1}}=\frac{1}{\sqrt{3}}$\\

\textbf{Ответ: A}

%5
\item
$\Var(2X-Y+4)= 2^2\Var(X)+\Var(Y)+2\Cov(2X,-Y)=2^2\Var(X)+\Var(Y)-2\cdot2\Cov(X,Y)=4\cdot12+1-4\cdot2=41$\\

\textbf{Ответ: E}

%6
\item
Ковариационная матрица для X и Y в общем виде:
\begin{center}
\begin{pmatrix}
\Var(X)  & \Corr(X,Y) \\
\Corr(X,Y) & \Var(Y) 
\end{pmatrix}
\end{center}\\
Так как ковариационная матрица является единичной, значит, она имеет вид:
\begin{center}
\begin{pmatrix}
1  & 0\\
0 & 1 
\end{pmatrix}
\end{center}\\
Следовательно, $\E(X)=\E(Y)=0, \Var(X)=\Var(Y)=1, \Cov(X,Y)=0$.\\

По свойству ковариации, если $\Cov(X,Y)=0$, то случайные величины X и Y являются независимыми.\\

\textbf{Ответ: D}

%7
\item 
$$\Corr(X+Y, 2Y-7)= \frac{\Cov(X+Y,2Y-7)}{\sqrt{\Var(X+Y)\cdot \Var(2Y-7)}}=\frac{\Cov(X,2Y-7)+\Cov(Y,2Y-7)}{\sqrt{\Var(X)+\Var(Y)+2\Cov(X,Y)\Var(2Y)}}$$
$$=\frac{2\Cov(X,Y)+2\Cov(Y,Y)}{\sqrt{\Var(X)+\Var(Y)+2\Cov(X,Y)4\Var(Y)}}=\frac{2\Cov(X,2)+2\Var(Y)}{\sqrt{\Var(X)+\Var(Y)+2\Cov(X,Y)4\Var(Y)}}$$\\

Так как по условию $\Corr(X,Y)=\frac{\Cov(X,Y)}{\sqrt{\Var(X)\Var(Y)}}=0,5$ и $\Var(X)=\Var(Y)$, то $\Corr(X,Y)= \frac{\Cov(X,Y)}{\sqrt{\Var(X)\cdot \Var(Y)}}=\frac{\Cov(X,Y)}{\Var(X)}=0,5$, следовательно, $\Cov(X,Y)=0,5\Var(X)$.\\

Тогда:\\
$$\Corr(X+Y, 2Y-7)=\frac{2\cdot 0,5\Var(X)+2\Var(X)}{\sqrt{[2\Var(X)+2\cdot 0,5 \Var(X)]\cdot 4\Var(X)}}=\frac{3\Var(X)}{\sqrt{12\Var(X)^2}}=\frac{\sqrt{3}}{2}$$\\

\textbf{Ответ: B}

%8
\item $$\P(0,2<\xi<0,7)=F(0,7)-F(0,2)$$

Так как при функция распределения при $x \in [0; 1]: F_\xi(x) = \frac{x-0}{1-0}=x$, то:\\
$$F(0,7) = 0,7; F(0,2) = 0,2$\\
Следовательно, $$\P(0,2 < \xi < 0,7) = 0,7 - 0,2 = 0,5$$\\

\textbf{Ответ: D}
%9
\item Согласно центральной предельной теореме, если случайные величины $\xi_1,\dots, \xi_n$  независимы и одинаково распределены, то:\\
$$\frac{S_n-\E[S_n]}{\sqrt{\Var(S_n)}} \xrightarrow d Z\sim \cN(0,1) \text{ при } n \rightarrow \infty$$\\
где $S_n = \xi_1 + \dots + \xi_n; \E[S_n] = n\cdotE(\xi_i) ; \Var[S_n] = n\cdot \Var[\xi_i]$\\

Тогда функция плотности у предела стандартизированных сумм имеет вид:$f_\xi(x)=\frac{1}{\sqrt{s\pi}}e^{-\frac{x^2}{2}}$\\

Следовательно, 
$$\lim_{n\to\infty}\P(\frac{S_n-\E[S_n]}{\sqrt{\Var(S_n)}}>1)=\int_{1}^{+\infty} f_\xi(t)dt = \int_{1}^{+\infty} \frac{1}{\sqrt{2\pi}}e^{-\frac{t^2}{2}}dx$$\\
\textbf{Ответ: B}

%10
\item
$\E(X_i) = 400, \Var(X_i) = 400, n = 100$. Так как $S_n = X_1 + \dots + X_n$, следовательно:\\
$$\E(S_n) = n\cdot \E(Х_i) = 40000$$
$$\Var(S_n) = n\cdot \Var(Х_i) = 40000$$
Согласно ЦПТ:\\
$$\Р(S_n > 40400) = \P(Z > \frac{40400  - 40000}{\sqrt{40000}} = \P(Z > 2) = 1 - \P(Z < 2) = 1 - 0,97725 = 0,022750132$$\\
\textbf{Ответ: D}

%11
\item $\E(X_i) = 10000$. Согласно неравенству Маркова:\\
$$\P(X_i \geqslant 50000) \leqslant \frac{E(X_i)}{50000}$$
$$\P(X_i \geqslant 50000) \leqslant  \frac{10000}{50000} = 0,2$$\\
\textbf{Ответ: A}

%12
\item События являются несовместными, если они не содержат общих исходов, то есть наступление одного из этих событий исключает наступление другого.\\

События А и С являются несовместными, так как если произойдет С (все три раза выпал орёл), то А не произойдет (решка не может выпасть хотя бы 1 раз).\\

Так как монетка подбрасывается три раза, то один раз может выпасть решка и другой раз – орел, следовательно, А и В являются совместными.\\

\textbf{Ответ: A}

%13
\item $\E(X_i) = 50000, \sqrt{\Var(X_i)} = 10000$. Согласно неравенству Чебышева:\\
$$\P=(|X_i - \E(X_i)| \leqslant 20000) \geqslant 1 -  \frac{\Var(X_i)}{20000^2} = 1 - \frac{10000^2}{20000^2} = 1 - 0,25 = 0,75$$\\
\textbf{Ответ: C}

%14
\item
\begin{equation*}
\xi_i = 
 \begin{cases}
   1, &\text{с вероятностью $p=0,6$}\\
   0, &\text{с вероятностью $1-P=0,4$}\\
 \end{cases}
\end{equation*}\\

Следовательно, $\xi_i^{2016} = \xi_i$, значит, $\frac{\xi_1^{2016} +…+ \xi_n^{2016}}{n} = \frac{\xi_1 +…+ \xi_n}{n} = \overline{\xi} = \frac{(0,6\cdot1 +0,4\cdot0)}{n}=0,6$\\
Таким образом, $\plim_{n\to\infty} \frac{\xi_1+\dots+\xi_n}{n}=\plim_{n\to\infty} (0,6) = 0,6$\\

\textbf{Ответ: A}

%15
\item Пусть вероятность успеха (выпалo 6) в одном испытании равна $p = \frac{1}{6}$, тогда вероятность неудачи (выпалo не 6): $1 - p = \frac{5}{6}$.
\\Следовательно, по формуле Бернулли:$\P(X = k) = C_n^k\cdot \p^k (1-\p)^{n-k}$ (так как $k \in \{0, 1, 2,…, n\}), где n = 5, k = 2$\\
Тогда:\\
$$\P(X = 2) = C_5^2\cdot(\frac{1}{6})^2\cdot(\frac{5}{6})^3=\frac{5!}{2!\cdot3!}\cdot\frac{5^3}{6^5}=\frac{2\cdot5^4}{6^5}=\frac{2\cdot5^4}{2\cdot3\cdot6^4} =\frac{5^4}{2^4\cdot3^5}$$\\
\textbf{Ответ: нет равильного ответа}

%16
\item Так как вероятность успеха (выпалo 6) в одном испытании равна $p = \frac{1}{6}$, тогда вероятность неудачи (выпалo не 6): $1 - p = \frac{5}{6}$. Число бросков n = 5.\\

$$\E(X) = n\cdot p = 5\cdot\frac{1}{6}=\frac{5}{6}$$
$$\Var(X) = n\cdot p\cdot(1-p) = 5\cdot\frac{1}{6} \cdot \frac{5}{6} = \frac{25}{36}$$\\

\textbf{Ответ: нет равильного ответа}

%17
\item
Наиболее вероятное число шестерок считается как медиана m по формуле:\\
$$np - (1-p) \leqslant m \leqslant np + p$$\\
Число бросков n = 5.\\
Так как вероятность успеха (выпалo 6) в одном испытании равна $p = \frac{1}{6}$, тогда вероятность неудачи (выпалo не 6): $1 $–$ p = \frac{5}{6}$.\\
Следовательно, $$5\cdot\frac{1}{6} - \frac{5}{6} \leqslant m \leqslant 5\cdot \frac{1}{6} + \frac{1}{6}$$
$$ 0 \leqslant m \leqslant 1 $$\\
Значит, наиболее вероятное число шестерок 0 и 1\\
\textbf{Ответ: E}

%18
\item Условие:

\begin{center}
\begin{tabular}{lcccccc}
\hline
X=x_i      & 1  & 2  & 3 & 4 & 5 & 6  \\ \hline
P(X=x_i) &1/6 &1/6 &1/6 &1/6 &1/6 &1/6 \\ \hline
\end{tabular}
\end{center}\\
$\E(X) = \frac{1}{6}\cdot(1 + 2 + 3 + 4 + 5 + 6) = 3,5$ – математическое ожидания выпавших очков за 1 бросок.\\

Следовательно, математическое ожидание суммы выпавших очков за 5 бросков равно: 
$5\cdot E(X) = 5\cdot3,5 = 17,5$\\

\textbf{Ответ: E}

%19
\item Общий вид функции плотности двумерного нормального распределения:\\
$$f_{(\xi,\eta}(x,y)=\frac{1}{2\pi\sigma_x\sigma_y\sqrt{1-\rho^2}}e^{(\frac{-1}{2(1-\rho^2)})(\frac{(x-\mu_x)^2}{\sigma_x^2}-2\rho\frac{(x-\mu_x)(y-\mu_y)}{\sigma_x\sigma_y}+\frac{(y-\mu_y)^2}{\sigma_y^2})}$$\\
где $\rho=\Corr(\xi,\eta)=\frac{\Cov(\xi,\eta)}{\sigma_x\sigma_y}$\\
$$\xi \sim \cN(\mu_x,\sigma_x^2)$$
$$\eta \sim \cN(\mu_y,\sigma_y^2)$$\\
Так как по условию $\mu_x=\mu_y=0, \sigma_x^2=\sigma_y^2=1, \ov(\xi,\eta) = 0,5, то \rho= \Cov(\xi,\eta) = 0,5$, следовательно, функция плотности имеет вид:\\
$$f_{(\xi,\eta}(x,y)=\frac{1}{2\pi\sqrt{1-0,25}}e^{(\frac{-4}{2\cdot3})(x^2-xy+y^2)}=\frac{1}{\pi\sqrt{3}}e^{(-\frac{2}{3})(x^2-xy+y^2)}$$\\
Чтобы найти a и b, должны выполняться условия:\\
\begin{center}
 \begin{cases}
   $\pi\sqrt{3}=2\pi a$\\
   $-\frac{2}{3}(x^2-xy+y^2)$=$-\frac{1}{2a^2}(x^2-bxy+y^2)$\\
 \end{cases}
 \end{center}\\
Разделив первое уравнение на $2\pi$, получаем $a=\frac{\sqrt{3}}{2}$. Тогда второе уравнение примет вид:
$$-\frac{2}{3}(x^2-xy+y^2)=-\frac{2}{3}(x^2-bxy+y^2)$$
Значит, b = 1.\\

\textbf{Ответ: C}

%20
\item $$\E(\xi - 0,5\eta) = \E(\xi) - 0,5\E(\eta) = 0 - (0,5)\cdot0 = 0$$
$$\Var(\xi - 0,5\eta) = \Var(\xi) + 0,25\Var(\eta) - 2\cdot(0,5)\Cov(\xi, \eta) = 1 + 0,25\cdot 1 - 0,5 = 0,75$$
$$\Cov(\xi - 0,5\eta , \eta) = \Cov(\xi, \eta) - 0,5\cdot \Cov(\eta, \eta) = 0,5 - (0,5)\cdot1 = 0$$\\
Следовательно, компоненты вектора z некоррелированы и независимы (С и D неверно),\\
$$\xi - 0.5\eta \sim \cN(0; 0,75) \text{(Е неверно)}$$\\
А неверно, так как случайные величины зависимы\\
B верно, так как $z = (\xi - 0,5\eta; \eta)^T$ является двумерным нормальным вектором (распределение N \begin{pmatrix}0&0,75&0\\0&0&1\end{pmatrix}\\
\textbf{Ответ: B}

%21
\item
$$\E(\xi) = 0 = \E(\eta), \Var(\xi) = 1 = \Var(\eta), \Cov(\xi, \eta) = 0,5 \text{по условию}$$
$$\rho=\Corr(\xi,\eta)=\frac{\Cov(\xi,\eta}{\sigma_x\sigma_y}=\frac{0,5}{1\cdot1}=0,5$$
$$\E(\xi|\eta=1) = \E(\xi)+\rho\frac{\sqrt{\Var(\xi)}}{\sqrt{\Var(\eta)}}(1-\E(\eta))=0+(0,5)\cdot1\cdot(1-0)=0,5$$
$$\Var(\xi|\eta=1) = \Var(\xi)(1-\rho^2)=1\cdot (1-0,25)=0,75$$

\textbf{Ответ: C}

%22
\item Заполним таблицу на основе данных условия:\\
\begin{center}
\begin{tabular}{lcc}
\hline
X|Y=0      & 0  & 2   \\ \hline
\P(X|Y=y) & 1/2 & 1/2 \\ \hline
\end{tabular}
\end{center}\\
$$\P(X=0|Y=0)=\frac{\P(X=0\cap Y=0)}{\P(Y=0)}=\frac{\frac{1}{6}}{\frac{1}{3}}=0,5$$
$$\P(X=2|Y=0)=\frac{P(X=2\cap Y=0)}{\P(Y=0)}=\frac{\frac{1}{6}}{\frac{1}{3}}=0,5$$
$$\E(X|Y=0)=0\cdot0,5+2\cdot0,5=1$$
\textbf{Ответ: A}

%23
\item
$$\P(X=0|Y<1)=\P(X=0|Y=-1)+\P(X=0|Y=0)$$
$$\P(X=0|Y<1)=\frac{\P(X=0 \cap Y<1)}{\P(Y<1)}=\frac{0+\frac{1}{6}}{\frac{2}{3}}=0,25$$
\textbf{Ответ: B}

%24
\item
По свойству дисперсии: $\Var(Y) = \E(Y^2) - [\E(Y)]^2$\\
\begin{center}
\begin{tabular}{l|lll|l}
    & Y=-1 & Y=0 & Y=1 & \sum   \\ \hline
X=0 & \multicolumn{1}{c}{0}   & \multicolumn{1}{c}{1/6} & \multicolumn{1}{c|}{1/6} & \multicolumn{1}{c}{1/3} \\
X=2 &  1/3  &  1/6 &  1/6 &  2/3 \\ \hline
\sum   & \multicolumn{1}{c}{1/3} & \multicolumn{1}{c}{1/3} & \multicolumn{1}{c|}{1/3} & \multicolumn{1}{c}{1}  
\end{tabular}
\end{center}\\
$$\E(Y^2)=(-1)^2\cdot\frac{1}{3}+0\cdot\frac{1}{3}+1^2\cdot \frac{1}{3}=\frac{2}{3}$$
$$\E(Y)^2=((-1)\cdot\frac{1}{3}+0\cdot\frac{1}{3}+1^2\cdot \frac{1}{3})^2=0$$
Следовательно, $\Var(Y) = \frac{2}{3}$\\

\textbf{Ответ: A}

%24
\item
$\Cov(X, Y) = \E(XY) – \E(X)\E(Y)$\\
Условие:\\
\begin{center}
\begin{tabular}{l|lll|l}
    & Y=-1 & Y=0 & Y=1 & \sum   \\ \hline
X=0 & \multicolumn{1}{c}{0}   & \multicolumn{1}{c}{1/6} & \multicolumn{1}{c|}{1/6} & \multicolumn{1}{c}{1/3} \\
X=2 &  1/3  &  1/6 &  1/6 &  2/3 \\ \hline
\sum   & \multicolumn{1}{c}{1/3} & \multicolumn{1}{c}{1/3} & \multicolumn{1}{c|}{1/3} & \multicolumn{1}{c}{1}  
\end{tabular}
\end{center}\\
Составим таблицу на основе условия:\\
\vspace{2mm}
\begin{center}
\begin{tabular}{lccc}
\hline
XY      & -2 & 0 & 2   \\ \hline
\P(XY) & 1/3 & 1/3+1/6=1/2 & 1/6 \\ \hline
\end{tabular}
\end{center}\\
$$\E(XY) = (-2)\cdot\frac{1}{3} + 0 + 2\cdot\frac{1}{6}  = \frac{-1}{3}$$
$$\E(Y) = (-1)\cdot\frac{1}{3}+ 0\cdot\frac{1}{3}+ 1\cdot\frac{1}{3}= 0$$
Следовательно, $\Cov(X, Y) = \frac{-1}{3} $–$ 0 = \frac{-1}{3}$\\

\textbf{Ответ: B}

%26
\item
$$\int\limits_{0}^{0,5} {\int\limits_{0}^{0,5} 9x^2y^2\,dx}\,dy=9\int\limits_{0}^{0,5} \left.\frac{x^3}{3}\right|_0^{0,5} \,dy=3\int\limits_{0}^{0,5} \frac{1}{8}y^2 \,dy=\frac{3}{8}(\left.\frac{y^3}{3}\right|_0^{0,5})=\frac{1}{8}\cdot\frac{1}{8}=\frac{1}{64}$$
\textbf{Ответ: D}
\end{enumerate}

\section{Промежуточный экзамен 2017-2018 гг}
\begin{enumerate}

%1
\item $$\Var(X) = \E(X^2) - \E[(X)]^2 \Rightarrow  \E(X^2) = 6 + 4 = 10$$
По неравенству Маркова: \\
$$\P (X^2 \geqslant \alpha) \leqslant \frac {E(X^2)}{\alpha}$$
$$\P (X^2\geqslant 100) \leqslant \frac{10}{100} = 0.1$$
\textbf{Ответ: A}

%2
\item Распределение Пуассона \Rightarrow $ \E(\xi) = \lambda, \Var(\xi) = \lambda \Rightarrow \E(\xi^2) = \Var(\xi) + [\E(\xi)]^2= \lambda + \lambda^2 = \lambda(\lambda+1)$ \\

\textbf{Ответ: E}

%3
\item $$\Corr(X+Y, Y)= \frac {\Cov(X+Y,Y)}{\sqrt{Var(X+Y)\cdot \Var(Y)}}=\frac{\Cov(X+Y)+\Var(Y)}{\sqrt{[\Var(X)+\Var(Y)+2\Cov(X,Y)] \cdot \Var(Y)}}=\frac{-3+9}{\sqrt{9\cdot (9+4-6)}}=$$
$$\frac{6}{\sqrt{63}}=\frac{2}{\sqrt{7}}$$ \\
\textbf{Ответ: C}

%4
\item Стандартное нормальное распределение \Rightarrow $ \xi  \sim \mathcal{N}(0, 1)$ \Rightarrow  $ функция плотности: f_\xi = \frac{1}{\sqrt{2\pi}} e^\frac{-x^2}{2}$\\
$$\P(\xi \in [-1, 2]) = \int\limits_{-1}^{2} \frac{1}{\sqrt{2\pi}}e^\frac{-x^2}{2}\,dx$$
\textbf{Ответ: B}

%5
\item 
$$(X, Y) \sim \mathcal{U}$$
$$f_{X,Y} (1, 1)= \frac{S_\text{квадрата}}{S_\text{треугольника}} = \frac{1}{4}$$
\textbf{Ответ: A}

%6
\item По определению независимых событий в совокупности.\\

\textbf{Ответ: B}

%7
\item .
\begin{center}
$\xi \sim U[0; 4]$ \Rightarrow 
$F_\xi =$
 \begin{cases}
   \frac{\xi-0}{4-0} &\text{$\xi\in [0;4]$}\\
   0 &\text{$\xi<0$}\\
   1 &\text{$\xi>4$}
 \end{cases}
 \end{center}\\
$$\P(\xi \in [3; 6]) = \P(\xi \in [3; 4]) = 1-\P(\xi \in [0; 3]) = 1 - \frac{3-0}{4-0} = 1 - 0,75 = 0,25$$
\textbf{Ответ: B}

%8
\item X = {-5, …, 5}\\
Y = {-1, …, 1}\\
\begin{center}
\vspace{2mm}
\begin{tabular}{lccc}
\hline
Y      & -1  & 0   & 1   \\ \hline
P(Y=y) & 1/3 & 1/3 & 1/3 \\ \hline
\end{tabular}
\vspace{2mm}

\begin{tabular}{lcc}
\hline
Y^2      & 0  & 1   \\ \hline
\P(Y=y) & 1/3 & 2/3 \\ \hline
\end{tabular}\\
\vspace{2mm}

\begin{tabular}{lccc}
\hline
X      & -5  & \dots  & 5   \\ \hline
\P(Y=y) & 1/11 & \dots & 1/11 \\ \hline
\end{tabular}\\
\end{center}
\hangindent=22mm  \noindent \\
$$Y^2 = 0 \Rightarrow X = 2$$
$$Y^2 = 1 \Rightarrow X = 1$$
$$\P(X + Y^2 = 2) = \frac{1}{3} \cdot \frac{1}{11} + \frac{1}{11} \cdot \frac{2}{3} = \frac{1}{11}$$
\textbf{Ответ: C}

%9
\item 
$\text{П} = 1800 \Rightarrow \frac{\pi}{3} = 600  \Rightarrow 6 секторов$\\
$\P_\text{попасть в красный сектор} = \frac{1}{6}$ \\
\textbf{Ответ: E}

%10
\item 
$$\P(A\cap B) = 0.2$$
$$\P(A\cap B) = 0.6$$
$$\P(A \cup B) = \P(A) + \P(B) - \P(A \cap B) \Rightarrow \P(B) = 0.6 + 0.2 - 0.3 = 0.5$$
\textbf{Ответ: B}

%11
\item 
$$\Var(2X+Y+1) = 4 \Var(X) + \Var(Y) - 4 \Cov(X, Y) = 16 + 9 + 12 = 37$$
\textbf{Ответ: B}

%12
\item	
$$\E(X_i^2) = \Var(X_i) + [\E(X_i)]^2 = 1 + 0 = 1$$
По ЗБЧ:\\
$$ \p\lim_{x\to\infty}\frac{\sum X_i^2}{n}=\E(X_1^2)=1 $$
\textbf{Ответ: B}

%13
\item	
$$f_Y(y)=\int\limits_{0}^{1} 6xy^2\,dx = \left.{ 3x^2y^2}\right|_0^1=3y^2$$
 \begin{center}
$F_{X|Y} =$ 
 \begin{cases}
   \frac{f_{X,Y}(x,y)}{f_Y}=\frac{6x \cdot 0,25}{3 \cdot 0,25}=2x, &\text{$x \in [0;1]$}\\
   0, &\text{иначе}\\
 \end{cases}
 \end{center}
\textbf{Ответ: C}

%14
\item
$$\overline{X} = \frac{\sum X_i^2}{n}$$
$$\E(X) = 4, \Var(X) = 100$$
$$\E(\overline{X} = 4), \Var (\overline{X}) = \frac{100}{n} = 1 \Rightarrow \overline{X} \sim \mathcal{N}(4,1)$$
По ЦПТ:\\
$$\P(\overline{X}\leqslant5) = \frac{\overline{X}-4}{1} \leqslant \frac{5-4}{1} = \P(Z \leqslant 1) = 0.8413$$
\textbf{Ответ: C}

%15
\item 
$$\Cov(X + 2Y, 2X + 3) = 2\Var(X) + 4\Cov(X, Y) = 8 -12 = -4$$
\textbf{Ответ: A}

%16
\item 
$$\E((X-1)Y) = \E(XY - Y) = \E(XY) - \E(Y) = \Cov(X, Y) + \E(X) \cdot \E(Y) - \E(Y) = -3 - 2 - 2 = -7$$
\textbf{Ответ: B}

%17
\item .\\
\begin{center}
$X_i =$
 \begin{cases}
   1 &\text{$\p=\frac{1}{6}$}\\
   0 &\text{$\p=\frac{5}{6}$}
 \end{cases}
 \end{center}\\
$$X1 + X2 = 1 \Rightarrow X_1 = 1  \text{ и }  X_2 = 0 (\P = \frac{1}{6} \cdot \frac{5}{6} = \frac{5}{36}) \text{ или } X_1 = 0 \text{ и } X_2 = 1 (\P = \frac{1}{6} \cdot \frac{5}{6} = \frac{5}{36})$$
$$\frac{5}{36} + \frac{5}{36} = \frac{10}{36}$$
\begin{center}
\begin{tabular}{lcc}
\hline
X_i      & 0  & 1   \\ \hline
\P & 1/2 & 1/2 \\ \hline
\end{tabular}
\end{center}\\
\textbf{Ответ: B}

%18
\item 
$$\P(X + Y < 3)$$
Пусть $$Z = X + Y$$
Тогда $$\E(Z) = \E(X) + \E(Y) = 2 + 1 = 3$$
$$\Var(Z) = \Var(X) + \Var(Y) + 2\Cov(X,Y) = 7$$
ЦПТ:\\
$$\P(\frac{Z-3}{\sqrt{7}} < \frac{3-3}{\sqrt{7}}) = 
\P(Z < 0) = 0.5$$
\textbf{Ответ: C}

%19
\item 
$$\P_\text{честный|выпала 6} = \frac{\P_{\text{честный} \cap \text{выпала 6}}}{\P_\text{выпала 6}} \frac{\frac{3}{5} \cdot \frac{1}{6}}{\frac{3}{5} \cdot \frac{1}{6} + \frac{1}{5} \cdot \frac{1}{2} + \frac{1}{5} \cdot \frac{1}{10}} = \frac{5}{11}$$
\textbf{Ответ: C}

%20
\item 
По свойствам ковариационной матрицы: варианты D и E отпадают (не симметричные), A и B не подходят, т.к. отрицательно определены.\\

\textbf{Ответ: C}

%21
\item 
$$\E(\alpha X + (1 - \alpha)Y) = \alpha\E(X) +(1 - \alpha)\E(Y) = 0$$
$$-\alpha + (1 - \alpha)2 = 0$$
$$2 - 3\alpha = 0 \Rightarrow \alpha = \frac{2}{3}$$
\textbf{Ответ: A}

%22
\item
$$\P(\xi=0)=C_0^2\cdot(\frac{3}{4})^0\cdot(\frac{1-3}{4})^2=\frac{2!}{2!}\cdot1\cdot\frac{1}{16}=\frac{1}{16}$$
\textbf{Ответ: B}

%23
\item
$$\P(k \geqslant 1) = 1 - \P(k = 1) - \P(k = 0) = 1 -  \frac{4}{1!} \cdot e^{-4} \cdot \frac{-1}{0!} \cdot e^{-4} = 1-e^{-4}$$
\textbf{Ответ: C}

%24
\item 
$$\E(\xi^2) = \Var(\xi) + [\E(\xi)]^2 =\P(1 - \P) + \P^2 = \P - \P^2 + \P^2 = \P$$
$$\text{Бернулли} \Rightarrow \Var(\xi) = \P(1- \P), \E(\xi) = \P$$
\textbf{Ответ: B}

%25
\item
Экспоненциальное распределение $\Rightarrow \E(X) = \frac{1}{\lambda}, \Var(X) = \frac{1}{\lambda^2}$\\
$$\E(X^2) = \Var(X) + [\E(X)]^2 = \frac{1}{\lambda^2} + \frac{1}{\lambda^2} = \frac{2}{\lambda^2}$$
\textbf{Ответ: A}

%26
\item
$\frac{\pi}{3} = 600 \Rightarrow 6 \text{секторов}$\\
$$\P(A) = \P(B) = \frac{1}{6}$$\\
A) \P(A)+\P(B) = \frac{1}{3} \neq 1 \Rightarrow $не полная группа событий$\\

B) $\P(A) = \P(B) = \frac{1}{6}$\\

C) Они зависимы, т.к. не могут произойти враз\\

D) Вероятности событий равны\\

E) Они несовместны (не могут произойти враз)\\

\textbf{Ответ: E}

%27
\item
$$\E(XY)=\int\limits_{0}^{1} {\int\limits_{0}^{1} xy6xy^2\,dy}\,dx=\left.{\int\limits_{0}^{1} 2xy^3y^3}\right|_0^1\,dy=\int\limits_{0}^{1} 2y^3\,dy=\left.\frac{y^4}{2}\right|_0^1=\frac{1}{2}$$


%28
\item
$$\Var(\alpha X +(1-\alpha) Y)=\alpha^2 Var(X) + (1-\alpha)^2 \Var(Y) + 2\alpha\cdot(1-\alpha)\Cov(X,Y) = 19\alpha^2 - 24\alpha+9$$
$\alpha_\text{вершины}=\frac{24}{38}=\frac{12}{19}\\$

\textbf{Ответ:нет правильного ответа}

%29
\item
$\P_\text{без рюкзака}=\frac{50}{200}\cdot\frac{1}{2}+\frac{55}{200-50}\cdot\frac{150}{200}=\frac{1}{8}\cdot\frac{11}{40}=0,4$\\

\textbf{Ответ: B}

%30
\item
$\E(X)=2, \Var(X)=6$\\
По неравенству Чебышева:\\
$$\P(|2-X|\leqslant10)\qeqslant1-\frac{6}{100}=1-\frac{\Var(X)}{\varepsilon^2}=0,94$$
$$\P(|\E(X)-X|\leqslant10)=1-\P(|\E(X)-X|\leqslant10)$$
\textbf{Ответ: C}
\end{enumerate}

\section{Контрольная работа №2 (2018-2019гг)}
\be \textbf{Задача 6.}\\
А) $R_A = \frac{R_1}{2} + \frac{R_2}{2} + 0\cdot R_3$  - доходность портфеля A\\

$R_B = \frac{R_2}{2} + \frac{R_3}{2} + 0\cdot R_1$  - доходность портфеля B\\

$\E(R_A) = \frac{1}{2}\cdot5 + \frac{1}{2}\cdot10 = 7,5$  - ожидаемая доходность портфеля A\\

$\E(R_B) = \frac{1}{2}\cdot10 + \frac{1}{2}\cdot15 = 12,5$ - ожидаемая доходность портфеля В\\

Б) Найдем риски портфелей A и В.\\

$\Var(R_A) = \Var(\frac{R_1}{2} + \frac{R_2}{2}) = \Var(\frac{R_1}{2}) + \Var(\frac{R_2}{2}) + 2\Cov(\frac{R_1}{2}, \frac{R_2}{2}) = \frac{\Var(R_1)+\Var(R_2)}{4} + \frac{\Cov(R_1,R_2)}{2} = \frac{50}{4} + \frac{100}{4}+ \frac{20}{2} = 47,5$\\

$\Var(R_B) = \Var(\frac{R_2}{2} + \frac{R_3}{2}) = \Var(\frac{2_1}{2}) + \Var(\frac{R_3}{2}) + 2\Cov(\frac{R_2}{2}, \frac{R_3}{2}) = \frac{\Var(R_2)+\Var(R_3)}{4} + \frac{\Cov(R_2,R_3)}{2} = \frac{100}{4} + \frac{150}{4}- \frac{10}{2} = 57,5$\\

Так как риск портфеля равен $\sqrt{\Var(R_i)}$, то:\\

Риск портфеля $A = \sqrt{47,5} \approx 6,89$\\

Риск портфеля $B = \sqrt{57,5} \approx 7,58$\\

В) Портфель B имеет большую ожидаемую доходность, так как $\E(R_B) = 12,5 > \E(R_A) = 7,5$\\

Портфель А имеет меньший риск, так как $\sqrt{\Var(R_B)} \approx 7,58 > \sqrt{\Var(R_A)} \approx 6,89$\\

$$\Corr(R_A,R_B)=\frac{\Cov(R_A,R_B)}{\sqrt{\Var(R_A)\Var(R_B)}}=
\frac{\Cov(\frac{R_1}{2}+\frac{R_2}{2},\frac{R_2}{2}+\frac{R_3}{2})}{\sqrt{47,5}\cdot\sqrt{57,5}}=
\frac{\Cov(\frac{R_1}{2},\frac{R_2}{2}+\frac{R_3}{2})+\Cov(\frac{R_2}{2},\frac{R_2}{2}+\frac{R_3}{2})}{\sqrt{47,5}\cdot\sqrt{57,5}}=$$
$$\frac{\Cov(\frac{R_1}{2},\frac{R_2}{2})+\Cov(\frac{R_1}{2},\frac{R_3}{2})+\Cov(\frac{R_2}{2},\frac{R_2}{2})+\Cov(\frac{R_2}{2},\frac{R_3}{2})}{\sqrt{47,5}\cdot\sqrt{57,5}}=\frac{\frac{\Cov(R_1,R_2)}{4}+\frac{\Cov(R_1,R_3)}{4}+\frac{\Cov(R_2,R_2)}{4}\frac{\Cov(R_2,R_3)}{4}}{\sqrt{47,5}\cdot\sqrt{57,5}}=$$
$$\frac{\frac{20}{4}+\frac{-10}{4}\frac{Var(R_2)}{4}+\frac{-10}{4}}{\sqrt{47,5}\cdot\sqrt{57,5}}=
\frac{\frac{100}{4}}{\sqrt{47,5}\cdot\sqrt{57,5}} \approx 0,48$$

Д) Чтобы собрать собственный портфель $R_c$, обладающий доходностью не меньшей, чем портфели A и B, но меньшим риском, нужно минимизировать риски (т.е. нужно решить задачу минимизации $\Var(R_с)$) при условии, что сумма долей ценных бумаг равна 1:\\

\left\{\begin{array}{l}
\[$\min \Var(R_c)=50w_1^2+100w_2^2+150w_3^2+40w_1w_2-20w_1w_3-20w_2w_3$\\
s.t. $w_1+w_2+w_3=1$
\end{array}\right.\\

Составим вспомогательную функцию Лагранжа:\\

$L(w,\lambda)=50w_1^2+100w_2^2+150w_3^2+40w_1w_2-20w_1w_3-20w_2w_3+\lambda(w_1+w_2+w_3-1)$\\ 

Необходимым условием экстремума функции Лагранжа является равенство нулю ее частных производных по переменным $w_i$ и неопределенному множителю $\lambda$.\\

Составим систему:\\

\left\{\begin{array}{l}
\frac{\partial L}{\partial w_1}=100w_1+40w_2-20w_3+\lambda=0\\
\frac{\partial L}{\partial w_2}=40w_1+200w_2-20w_3+\lambda=0\\
\frac{\partial L}{\partial w_3}=-20w_1-20w_2+300w_3+\lambda=0\\
\frac{\partial L}{\partial \lambda}=w_1+w_2+w_3-1=0
\end{array}\right.\\

\left\{\begin{array}{l}
100w_1+40w_2-20w_3=-\lambda\\
40w_1+200w_2-20w_3=-\lambda\\
-20w_1-20w_2+300w_3=-\lambda
\end{array}\right.\\

$100w_1+40w_2-20w_3=40w_1+200w_2-20w_3=-20w_1-20w_2+300w_3$\\
$5w_1+2w_2-w_3  = 2w_1+10w_2-w_3  = -w_1-w_2+15w_3$ (уравнения 1,2 и 3 соответсвенно)\\

Выразим из 1 и 2 уравнений $w_2=\frac{3w_1}{8}$ и из 2 и 3 $w_3=\frac{11w_2+3w_1}{14}$, подставим $w_2$ в $w_3$ и получим $w_3=\frac{19\cdot3w_1}{14\cdot8}\\

Подставим выраженные через $w_1$ уравнения в ограничение системы:\\

$w_1+\frac{3w_1}{8}+\frac{19\cdot3w_1}{14\cdot8}=1$\\

Приводим подобные слагаемые в левой части уравнения:\\

$\frac{211w_1}{112}=1$\\

Следовательно, $w_1=\frac{112}{211}, w_2=\frac{42}{211}, w_3=\frac{57}{211}$\\

$\Var(R_c)=29,27944$\\

$\sqrt{Var(R_c)} = 5,411048$ – риск портфеля С\\

$\E(R_c) = \frac{112}{211}\cdot5 + \frac{42}{211}\cdot10 + \frac{57}{211}\cdot15 = 8,696682$ - ожидаемая доходность портфеля С\\

Так как риск портфеля С меньше рисков портфелей А и В, а ожидаемая доходность портфеля С больше ожидаемой доходности портфеля А (то есть не имеет не меньшую доходность, чем портфели А и В), то найденный портфель С с долями ценных бумаг \\

$w_1=\frac{112}{211} ,w_2=\frac{42}{211}, w_3= \frac{57}{211}$ является искомым.


\section{Контрольная работа №3 (2018-2019гг)}
\be \textbf{Задача 8.} \\
а)$\E(\hat{\theta_n})=\theta$ условие несмещенности\\

$\E(\hat{\theta_n})=\E(\frac{3}{2}\bar{X})=\frac{3}{2}\cdot\mathbb{E}(\frac{\sum X}{n}=\frac{3}{2}\cdot\frac{1}{n}\cdot n\E(X_i)=\frac{3}{2}\cdot\frac{2}{3}\cdot\theta=\theta$ \Rightarrow $несмещенная$\\
$\E(X^2)=\int\limits_{0}^{\theta} \frac{2x^2}{3\theta^2} \,dx = \left.\frac{2x^3}{3\theta^2}\right|_0^\theta = \frac{2}{3}\theta$\\
б) $\E(X^2)=\int\limits_{0}^{\theta} \frac{2x^3}{4\theta^2}\, dx = \left.\frac{2x^4}{\theta^2}\right|_0^\theta=\frac{\theta^2}{2}-\frac{4}{9}\theta^2=\frac{\theta^2}{18}$\\
$\Var(\hat{\theta_n}) = \Var(\frac{3}{2}bar{X})=\frac{9}{4n}\Var(X_i)=\frac{9}{4n}\cdot\frac{\theta^2}{18}=\frac{\theta^2}{8n}$\\

в) Так как оценка несмещенная, то для установления состоятельности достаточно одного условия:\\
$\lim_{n\to\infty}Var(\hat{\theta_n})=0$\\
$\lim_{n\to\infty}\frac{\theta^2}{8n}=0$ \Rightarrow $оценка состоятельна$\\

г) $\E(max(X_1,\dots,X_n)=\E(X_\textit{max})=\E(X_i)=\frac{2}{3}\theta \neq \theta$ \Rightarrow $смещенная$\\
$Величина смещения: x=$\frac{2}{3}$\theta-\theta=\frac{1}{3}\theta$\\

д) $MSE(\hat{\theta_n}$)=$\E($\frac{3}{2}-\theta$)^2 = \E(\frac{9}{4}\bar{X}^2 - 3\bar{X}\theta+\theta^2) = \frac{9\theta^2}{4\cdot2}-3\theta\cdot\frac{2}{3}\theta+\theta^2=\frac{\theta^2}{8}$\\
$MSE(\hat{T})=\E(max(X_1,\dots,X_n)-\theta)^2=\E(X_\textit{max}-\theta)^2=\E(X_\textit{max}^2)-2\theta \E(X_\textit{max}) +\theta^2=\frac{\theta^2}{2}-\frac{4}{3}\theta+\theta^2=\frac{1}{6}\theta^2$\\
$MSE(\hat{\theta_n})<MSE(\hat{T})$ $\Rightarrow \hat{\theta_n}$ эффективнее\\

\be \textbf{Задача 9.}\\
а) $f(x,\theta) = \frac{1}{\sqrt{2\pi\theta}}e^{\frac{-x^2}{2\theta}}$\\
$L=\prod_{i=1}^{n} f(x,\theta) = \prod_{i=1}^{n} \frac{1}{\sqrt{2\pi\theta}}e^{\frac{-x^2}{2\theta}} = ({\frac{1}{\sqrt{2\pi}}})^n \cdot ({\frac{1}{\sqrt{\theta}}})^n \cdot e^{-\frac{\sum X_i^2}{2\theta}}$\\
$ln L = -\frac{n}{2}ln(2\pi)+(-\frac{n}{2}ln\theta)-\frac{\sum X_i^2}{2\theta}$\\
$\frac{\partial lnL}{\partial\theta}=-\frac{n}{2}\cdot\frac{1}{\theta}-\frac{\sum X_i^2}{2}\cdot (-\frac{1}{\theta^2})=0$\\
$\hat\theta_\textit{ML} = \frac{\sum X_i^2}{n}$\\
$\frac{\partial^2 lnL}{\partial\theta^2} = \frac{n}{2\theta^2}-\frac{\sum X_i^2}{\theta^3} = \hat\theta_\textit{ML} = \frac{n^3}{2{(\sum X_i^2)}^2} - \frac{n^3}{\sum X_i^2} = \frac{n^3(1-2\sum X_i^2)}{2{(\sum X_i^2})^2} < 0$ \Rightarrow $max$\\

б) $\E(\frac{\sum X_i^2}{n})$=$\frac{1}{n}\E(\sum X_i^2)$ = $\frac{1}{n}$ \cdot n \cdot \E(X_i^2) = \E(X_i^2) = \Var(X_i) + {[\E (X_i)]}^2 = \theta + 0 = \theta$ \Rightarrow $оценка несмещена$\\

в) I(\theta) = -\E(\frac{\partial^2 lnL}{\partial\theta^2}) = -\E(\frac{n}{2\theta^2} - \frac{\sum X_i^2}{\theta^3}) = \frac{-n}{2\theta^2} + \frac{1}{\theta^3} \cdot n \cdot \E(X_i^2) = \frac{-n}{2\theta^2} + \frac{n}{\theta^2} = \frac{n}{2\theta^2}$\\


г) По неравенству Рао-Крамера: \\
$\Var(\hat{\theta_n}) = \Var(\frac{\sum X_i^2}{n}) = \frac{1}{n^2} \cdot \Var (\su X_i^2) = \frac{1}{n^2}\cdot n \cdot \Var(X_i^2) = \frac{1}{n} \Var(X_i^2) = \frac{1}{n}(\E(X^4) - [\E(X^2)]^2) = \frac{1}{n} \cdot (3\theta^2 - \theta^2) = \frac{2\theta^2}{n}$ \Rightarrow $\Var($\hat{\theta_n}$) = $\frac{1}{I(\theta)} \Rightarrow \text{оценка эффективная}$

\section{Контрольная работа №3 (2013-2014гг)}
\be \textbf{Задача 1.}\\
B_1 = 3\\
B_2 = 1\\
$\alpha = 4 \Rightarrow 1 - \alpha = 0,96$\\
$X_i \sim \mathcal{N}(\mu, \sigma^2)$\\
n = 13\\
$\overline{X}=51$\\
$\hat\sigma^2=3$\\

Доверительный интервал для мат. ожидания:\\
$$\P(\overline{X}-t_{12;\frac{\alpha}{2}}\cdot\frac{\hat\sigma}{\sqrt{n}}<\mu<\overline{X}+t_{12;\frac{\alpha}{2}}\cdot\frac{\hat\sigma}{\sqrt{n}})=1-\alpha$$
$$51-3,055\cdot\frac{\sqrt{3}}{\sqrt{13}}<\mu<51+3,055\cdot\frac{\sqrt{3}}{\sqrt{13}}$$
$$(49.5326; 52,4674)$$

Доверительный интервал для дисперсии:\\
$$\P(\frac{\hat\sigma^2(n-1)}{\chi^2_{n-1;1-\frac{\alpha}{2}}}<\sigma^2<\frac{\hat\sigma^2(n-1)}{\chi^2_{n-1;\frac{\alpha}{2}}})=1-\alpha$$
$$\frac{12\cdot3}{28,3}<\sigma^2$$
$$1,2721<\sigma^2$$

\be \textbf{Задача 2.}\\
$n_1 = 85$\\

$n_2 = 100$\\

$\hat p_1 = \frac{1}{3+1}=\frac{1}{4}$\\

$\hat p_2 = \frac{1}{1+1}=\frac{1}{2}$\\

$H_0: p_1 = p_2 = p_0$\\

$H_1: p_1 \neq p_2$\\

$p_0=\frac{85\cdot0,25+100\cdot}{185}$\\

$Z_p=\frac{0,25-0,5}{\sqrt{0,385\cdot(1-0,385)}}\sim \mathcal{N}(0,1)$\\

$Z_p=\frac{0,25-0,5}{\sqrt{0,385\cdot(1-0,385)}}\approx 3,483$\\

$H_0$ не отвергается в интервале (-2,05; 2,05). Так как расчётное значение не входит в этот интервал \Rightarrow $H_0$ отвергается \Rightarrow $преподаватели предъявляют разные требования$
$$p_\textit{value}=2\P(Z >|Z_\p|) = 2\P(Z > 3,483) = 2(1 - \P(Z \leqslant 3,483)) = 2(1 - 0,9997) = 0.0006$$


\section{Контрольная работа №3 (2009-2010гг)}
\be \textbf{Задача 1.}\\
$\overline{X}=\frac{-1,5+2,6+1,2-2,1+0,1+0,9}{6}=0,2$\\
$\hat\sigma^2=\frac{\sum {(X_i-\overline X)}^2}{n}=\frac{15,44}{6}=2,573$ неисправленная\\
$\hat\sigma^2=\frac{\sum {(X_i-\overline X)}^2}{n-1}=\frac{15,44}{5}=3,088$ исправленная\\
Вариационный ряд: $-2,1; -1,5; 0,1; 0,9; 1,2; 2,6$

\be \textbf{Задача 3.}\\
Чтобы оценка среднего по генеральной совокупности с наименьшей дисперсией была несмещенной, необходимо «взвесить» страты:\\

$\hat X_s=\sum_{i=1}^{s}w_i\overline{X}_i=0,1X_1 + 0,3X_2 + 0,6X_3$\\

$\Var(\hat X_i)=\frac{\sigma_i^2}{n_i}, где i $– номер страты$, S $– количество страт$\\

Решаем задачу:\\

\left\{\begin{array}{l}
\[\max \Var(\hat X_s)=\sum_{i=1}^{s} \frac{\sigma_i^2}{n_i}w_i^2\\
s.t. TC = \sum_{i=1}^{s} c_in_i
\end{array}\right.\\

\left\{\begin{array}{l}
\Var(\hat X_s)=(0,1)^2\frac{50^2}{n_1}+(0,3)^2\frac{20^2}{n_2}+(0,6)^2\frac{10^2}{n_3} \rightarrow min(n_1,n_2,n_3 )\\
s.t. 60000=150n_1+40n_2+15n_3
\end{array}\right.\\

$L = \frac{25}{n_1} +\frac{36}{n_2} +\frac{36}{n_3}+\lambda(150n_1+40n_2+15n_3  - 60000)$
\left\{\begin{array}{l}
\frac{\partial L}{\partial n_1}=-\frac{25}{n_1^2}+150\lambda=0\\
\frac{\partial L}{\partial n_2}=-\frac{36}{n_2^2}+40\lambda=0\\
\frac{\partial L}{\partial n_3}=-\frac{25}{n_3^2}+15\lambda=0
\end{array}\right

\left\{\begin{array}{l}
\frac{1}{6n_1^2}=\lambda\\

\frac{9}{10n_2^2}=\lambda\\

\frac{12}{5n_3^2}=\lambda
\end{array}\right.\\

\frac{1}{6n_1^2}=\frac{9}{10n_2^2}=\frac{12}{5n_3^2}\\

$n_2 = \sqrt{5,4}n_1$\\

$n_3 = \sqrt{14,4}n_1$\\
Подставим в ТС: $60000=150n_1+40\sqrt{5,4}n_1+15\sqrt{14,4}n_1$\\

$n_1^* = \frac{60000}{300} = 200; n_2^* \approx 465; n_3^* \approx 759$\\

Проверка (подставим найденные значения в бюджетное ограничение ТС): $150\cdot200 + 40\cdot465 + 15\cdot759 = 59985 \leqslant 60000$, следовательно, найдены верные значения.

\be \textbf{Задача 4.}\\
Так как выборка $X_1, X_2, …, X_n$ имеет равномерное распределение $\U[0; \theta]$, то:\\
$X_1 = X_2 = … = X_n = X_i$\\
$\E(X_i) = \frac{0 + \theta }{2} = \frac{\theta}{2}$\\
$\Var(X_i) = \frac{(\theta-0)^2}{12} = \frac{\theta^2}{12}$\\

Найдем оценку параметра θ методом моментов:\\ 
$\Е(Х_i) = \frac{\theta}{2}=\overline{X}$\\
$\hat\theta_\textit{MM}= 2\overline{X}$\\

Оценка является несмещенной, если выполняется: $E(\hat\theta) = \theta$.\\
$\E(\hat\theta_\textit{MM})=\E(2\overline{X})=2\E(\frac{\sum_{i=1}^{n} X_i }{n})=2\Е(Х_i) = 2\cdot \frac{\theta}{2} = \theta, следовательно, найденная оценка является несмещенной.$\\

Оценка является состоятельной, если она несмещенная (или асимптотически несмещенная) и $\lim_{n\to\infty}\Var(\hat{\theta_n})=0$\\
$\Var(\hat\theta_\textit{MM})=\Var(2\overline{X})=4\Var(\overline{X})=\frac{4\Var(X_i)}{n}=\frac{4}{n}\cdot\frac{\theta^2}{12}=\frac{\theta^2}{3n}$\\
$\lim_{n\to\infty}\frac{\theta^2}{3n}=0$, следовательно, оценка состоятельная\\

Оценка считается эффективнее, если ее дисперсия меньше. Так как $X_1, X_2,\dots, X_n$ – независимые одинаково распределенные случайные величины $X_1 = X_2 = … = X_n = X_i$, то $X_i = X_n = max(X_1,\dots, X_n)$.\\
$\Var(T)=\Var(\frac{n+1}{n}X_i)=({\frac{n+1}{n}})^2\Var (X_i)=({\frac{n+1}{n}})^2\cdot\frac{\theta^2}{12}$\\

Сравним дисперсии оценок:\\
$\Var(\hat\theta_\textit{MM})-\Var(T)=\frac{\theta^2}{3n}-({\frac{n+1}{n}})^2\cdot\frac{\theta^2}{12} = (4n-(n+1)^2 )\cdot\frac{\theta^2}{12n^2}=(4n-n^2-2n-1)\cdot\frac{\theta^2}{12n^2}=(n^2-2n+1)-\frac{\theta^2}{12n^2}=(n-1)^2-\frac{\theta^2}{12n^2}<0$, следовательно $\Var(\hat\theta_\textit{MM})<\Var(T))$, а значит $\hat\theta_\textit{MM}$ эффективнее.


\section{Контрольная работа №1 (2008-2009 гг)}
\be \textbf{Задача 3.}\\
$\P_\text{правда}=\frac{1}{3}$\\
Мэр либо сказал правду ($P=\frac{1}{3}$), либо соврал($\P=\frac{2}{3}$). Соответственно заместитель на следующий день также либо сказал правду ($\P=\frac{1}{3}$), либо соврал($\P=\frac{2}{3}$). Это два независимых события. Турнир состоится в двух случаях:\\
1) оба сказали правду;\\
2)мэр соврал, но на следующий день ситуация изменилась и заместитель сказал правду (его информация более актуальна).\\

Поэтому \P_\text{турнир состоится}=\frac{1}{3}\cdot\frac{1}{3}+\frac{2}{3}\cdot\frac{1}{3}=\frac{1}{3}

\section{Контрольная работа №2 (2010-2011гг)}
\be \textbf{Задача 6.}
20 пар – это 40 тапочек (генеральная совокупность N = 40). По условию, так как каждый размер представлен в двух парах, есть 4 пары одного размера. $n = 2$ ($Х_1$ и $X_2$, которые распределены одинаково).\\
$\P(X_1 = 36) = \frac{4}{40} = \frac{1}{10}$\\

\P(X_1 = x_i, X_2 = x_j)= 
 \begin{cases}
   \frac{4}{40}\cdot\frac{3}{39}\approx0,01 &\text{$i=j$}\\
   \frac{4}{40}\cdot\frac{4}{39}\approx0,008 &\text{$i\neqj$}\\
 \end{cases}\\
 
$Если i \neq j: \P(X_1 = 36, X_2 = 37) = \frac{4}{40}\cdot \frac{4}{39} \approx 0,01, $где $ \frac{4}{40} $– вероятность достать Х_1 первым$, а $\frac{4}{39}$ – вероятность достать Х_2 вторым.$\\

$Если i = j: \P(X_1 = 36, X_2 = 36) = \frac{4}{40}\cdot \frac{4}{39} \approx 0,008$\\
\vspace{2mm}
\begin{tabular}{lccccccccccc}
\hline
X_1,X_2      & 36  & 37  & 38 & 39 & 40 & 41 & 42 & 43 & 44 & 45  \\ \hline
\P            & 1/10 & 1/10 & 1/10 & 1/10 & 1/10 & 1/10 & 1/10 & 1/10 & 1/10 & 1/10  \\ \hline
\end{tabular}
\vspace{2mm}\\

$\E(\frac{X_1+X_2}{2})=\E(X_i)=\E(X_1)=\E(X_2)=\frac{36+\dots+45}{10}=40,5$\\

$\E(X_i^2)=\E(X_1^2)=\E(X_2^2)=\frac{36^2+\dots+45^2}{10}=1648,5$\\

$\Var(X_i)=E(X_i^2)-{E(X_i)}^2 = 1648,5 - 1640,25 = 8,25$\\

$\Var(\frac{X_1+X_2}{2})=\Var(\overline X})=\frac{\sigma^2}{n}(1-\frac{n-1}{N-1})=\frac{8,25}{2}(1-\frac{1}{39})\approx4,01$\\

$\Cov(X_1, X_2)=-\frac{\sigma^2}{N-1}=-\frac{8,25}{39}\approx-0,21$

\end{document}

